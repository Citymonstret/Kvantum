\documentclass[12pt]{article}

\usepackage{fullpage}
\usepackage{fancyhdr}
\usepackage{float}
\usepackage{caption}
\usepackage{array}
\usepackage{booktabs}
\usepackage[a4paper,margin=3cm,headsep=24pt, headheight=2cm]{geometry}
\usepackage[nottoc, notlot, notlof]{tocbibind}
\usepackage{minted}
\usepackage{hyperref}
\hypersetup{
    colorlinks,
    citecolor=black,
    filecolor=black,
    linkcolor=black,
    urlcolor=black
}

\pagestyle{fancy}

\newcommand{\version}{0.0.1}
\renewcommand{\baselinestretch}{1.5}
\renewcommand{\listingscaption}{Code Snippet}
\renewcommand{\listoflistingscaption}{List of Code Snippets}

\fancyhf{}
\lhead{IntellectualServer Manual}
\rhead{(Version: \version)}
\rfoot[C]{Page \thepage}

\begin{document}

\begin{titlepage}
\linespread{1}
\begin{center}
	\vspace*{1cm}
	\line(1,0){400}\\
	\vspace*{0.5cm}
	\Large\textbf{IntellectualServer} \\
	\Large{Manual} \\
	\line(1,0){400}\\
	\vfill
	Written for version: \version \\
	Last Revision: \today
	\vfill
\end{center}
\end{titlepage}
\clearpage

\tableofcontents
\clearpage

\section{Introduction}\label{sec:introduction}
IntellectualServer is a lightweight, innovative, portable and embeddable web server.
The server is written entirely in Java (Java 8, to be exact).
IntellectualServer is meant to be as versatile as it could possibly be, and it is intended to be used both as a library and as a standalone application.

This manual aims to provide the information necessary to \textbf{A:} install the web server as a standalone application, and \textbf{B:} use IntellectualServer in your own project.

\newpage

\section{Development}\label{sec:development}
This section aims to provide further insight into how IntellectualServer is built, and how it can be implemented into programs.
The library can be used as an embedded web server or a customizable standalone application.
IntellectualServer also has a plugin system, that allows you to load JARs as plugins during runtime.
This allows you to customize and extend IntellectualServer, without modifying the main executable.

\subsection{Getting Started}\label{subsec:gettingStarted}
It is highly recommended that you use a build manager when working with IntellectualServer.
IntellectualServer uses JitPack\footnote{https://jitpack.io} as its Maven repository.

\subsubsection{Artifacts}
IntellectualServer distributes a couple of different artifacts.
This manual will focus on the API and the Implementation.

Use:
\begin{itemize}
	\item ServerAPI, if you want to develop plugins or create your own server implementation
	\item Implementation, if you want to work with the standalone server or use the embedded server
\end{itemize}



\subsubsection{Maven}
You need to add JitPack to your repository list.
This can be done as follows:
\begin{listing}[H]
\caption{Add maven repository}
\begin{minted}{xml}
<repositories>
    <repository>
        <id>jitpack.io</id>
        <url>https://jitpack.io</url>
    </repository>
</repositories>
\end{minted}
\end{listing}

Then you add the IntellectualServer artifact of your liking.
The following example will load the Implementation:
\begin{listing}[H]
\caption{Add maven dependency}
\begin{minted}{xml}
<dependency>
    <groupId>com.github.IntellectualSites.IntellectualServer</groupId>
    <artifactId>Implementation</artifactId>
    <version>x.y.z</version>
</dependency>
\end{minted}
\end{listing}

\subsubsection{Gradle}
You need to add JitPack to your repository list.
This can be done as follows:
\begin{listing}[H]
\caption{Add Gradle repository}
\begin{minted}{groovy}
allprojects {
	repositories {
		...
		maven { url 'https://jitpack.io' }
	}
}
\end{minted}
\end{listing}

Then you add the IntellectualServer artifact of your liking.
The following example will load the Implementation:
\begin{listing}[H]
\caption{Add Gradle dependency}
\begin{minted}{groovy}
dependencies {
	compile group: 'com.github.IntellectualSites.IntellectualServer',
		name: 'Implementation', version: 'x.y.z'
}
\end{minted}
\end{listing}

\newpage
\listoflistings

\end{document}
